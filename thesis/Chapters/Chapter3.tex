
\chapter{Related Work}
\label{sec:related_work}


This chapter presents a variety of projects and scientific work related to sentiment analysis and the usage of entity based approaches for sentiment classification. The constant publication of opinionated data in social media networks provides an ever-growing source of valuable information for interested companies. Therefore, Sentiment Analysis has become a very popular research field in the scientific community. Pang and Lee ~\cite{pang2008opinion} in 2008 presented a survey that covers techniques and approaches for polarity sentiment classification and subjectivity identification. Moreover, in 2012 Bing Liu ~\cite{liu2012sentiment} published one of the most cited books related to sentiment analysis, where he provides a very complete survey of most relevant research topics related to opinion mining. 

Based on aforementioned literature and additional related works, the next section discusses a variety of sentiment analysis projects and the different approaches used in them. Finally, this chapter presents a few research works associated with entity-based sentiment classification and compares these techniques with the methods utilized and developed in this master's thesis.

\section{Sentiment Analysis}


The applications of Sentiment analysis are many, some of them include the classification of forum posts, blogs, news, product reviews and social network content. Therefore, because this thesis project is based on social media analysis, specifically Twitter data. The related work presented in this section is mainly focused on document-level sentiment analysis of tweets.

\pagebreak

The term of sentiment analysis was first introduced by Nasukawa and Yi in 2003 ~\cite{nasukawa2003sentiment}, in their work they extracted sentiments associated with polarities of positive or negative for specific subjects using semantic analysis with a syntactic parsing method. However, research about opinions and sentiments expressed in text appeared in 2001 where Das et al.~\cite{das2001yahoo} and Tong~\cite{tong2001operational} published their work about the analysis of market sentiment. 

It was not until 2009 where Bhayani et al.~\cite{go2009twitter} presented the first relevant research related to the usage of Twitter data for sentiment analysis. Bhayani et al. used a novel approach for automatic polar-classification of tweets where messages are classified as either positive or negative. Based on distant supervision and machine learning algorithms, Bhayani et al. generated a training corpus evaluating the presence of positive or negative emoticons such as ":D" or ":(" in each tweet. With this method, they managed to achieve an accuracy above 80\% for a polarity classification task.

In 2010 Pak and Paroubek~\cite{pak2010twitter} built a sentiment classifier that is able to determine positive, negative and neutral sentiments of English tweets. Using a distant supervision approach similar to Bhayani et al.~\cite{go2009twitter} for the generation of a training corpus, they implemented a multinomial Naive Bayes classifier extracting POS-tags and unigrams as binary features. In their results they showed higher precision by using a term presence rather than its frequency. Also, an increased accuracy was obtained by the usage of unigrams instead of bi-grams or three-grams. On the other hand, Barbosa and Feng in same year~\cite{barbosa2010robust} presented a 2-step sentiment analysis classification method which first classifies tweets as subjective and objective (neutral and polar), followed by a polarity classification of subjective tweets as positive or negative. Based on support vector machines, this 2-step approach proved an increased accuracy in comparison with single step classifiers. 

In 2011 Kouloumpis et al.~\cite{kouloumpis2011twitter} explored the utility of linguistic features
for the identification of sentiment in Twitter messages. This paper presents an extended distant supervision approach for generation of training data, the methods used consist on the inclusion of Twitter hashtags such as \textit{"\#bestfeeling, \#epicfail, \#news, etc."} to enhance the training data quality and include a third class to the classifier (neutral). Additionally, Kouloumpis et al. implemented a 3-step prepossessing method composed by the following steps: (1)tokenization, (2)normalization, (3)part-of-speech tagging. According to Kouloumpis et al.'s results this prepossessing stage improves the quality of the extracted features. In contrast to Kouloumpis et al., the methods used by Paltoglou and Thelwall~\cite{paltoglou2012twitter} in 2012 explored an unsupervised lexicon-based approach that predicts the level of emotional intensity contained in tweets. According to Paltoglou and Thelwall, this approach may be used for subjectivity identification and sentiment classification tasks, obtaining results comparable to state-of-the-art machine learning based methods.


In 2013 Saif et al.~\cite{MohammadKZ2013} developed a state-of-the-art support vector machine (SVM) classifier which obtained the best results in SemEval 2013 Twitter analysis task. SemEval (Semantic Evaluation) is an international competition for evaluations of computational semantic analysis systems. Many scientist and students from all around the world participated in this event, but Saif et al.'s sentiment classification approach excelled in its category. The classifier uses a large set of features to train a SVM, features such as n-grams, POS-Tags, hashtags, lexicons, emoticons, elongated words are just a part of the full set. However, the addition of negation context handling was one of the determinant factors to outperform the other competitors. Finally, the methods presented in this master's thesis for document-level features extraction are highly influenced by Saif et al.'s work.

\section{Entity Based Sentiment Analysis}

Currently, there are not many research works related to entity based sentiment analysis in Twitter. Therefore, this section intends to discuss the most relevant publications that explore the idea of an entity-centric sentiment classifier. Starting with Ding et al.~\cite{ding2008holistic}, in 2008 they presented a holistic lexicon-based approach to perform topic-based opinion mining. In their work, they determined the sentiment orientations (positive, negative or neutral) of opinions expressed in product features in reviews. With the usage of unsupervised lexicon-based techniques, Ding et al. defined a set of linguistic rules to extract opinion phrases expressed towards specific products in a given review text. Although Ding et al.'s approach achieved high accuracy for topic-based sentiment analysis on product reviews, this method may not perform just as well with noisy text such as tweets. Khoo et al.~\cite{thet2010aspect} in 2010 implemented an aspect-based sentiment classifier which was capable of extracting both sentiment orientation and sentiment strength of movie reviews. He considered the sentiment expressed towards different aspects of these movies. These aspects can be seen as sentiment targets, using sentence oriented linguistic clauses, Khoo et al. managed to obtain highly precise sentiment scores for movie aspects. However, one of the drawbacks of Khoo et al.'s solution is the absence of a neutral class on their approach. 



The most relevant scientific work related to this master's thesis was developed by Jiang et al.~\cite{jiang2011target} in 2011. They focused on target-dependent Twitter sentiment
classification which means that given an input query, they classify the sentiment orientation of the tweets as positive, negative or neutral based on the presence of positive, negative or neutral sentiments towards that query. In Jiang et al.'s approach they implemented a two-step SVM classifier incorporating
target-dependent and target-independent features. Additionally, they included related tweets (mentions and replies of each tweet) in the analysis. According to their experimental results, the two-steps methodology greatly improves the accuracy of entity-based sentiment classifiers. Nevertheless, because of the complexity of the methods and the necessity of analyzing related tweets in each document, the performance time is compromised and not suitable for real-time systems.

To conclude this chapter is important to clarify that this master's thesis is highly influenced by aforementioned research works. However, the specific combination of methods and techniques used in this project are not documented in any other related publication.   

